% Options for packages loaded elsewhere
\PassOptionsToPackage{unicode}{hyperref}
\PassOptionsToPackage{hyphens}{url}
%
\documentclass[
]{article}
\usepackage{amsmath,amssymb}
\usepackage{iftex}
\ifPDFTeX
  \usepackage[T1]{fontenc}
  \usepackage[utf8]{inputenc}
  \usepackage{textcomp} % provide euro and other symbols
\else % if luatex or xetex
  \usepackage{unicode-math} % this also loads fontspec
  \defaultfontfeatures{Scale=MatchLowercase}
  \defaultfontfeatures[\rmfamily]{Ligatures=TeX,Scale=1}
\fi
\usepackage{lmodern}
\ifPDFTeX\else
  % xetex/luatex font selection
\fi
% Use upquote if available, for straight quotes in verbatim environments
\IfFileExists{upquote.sty}{\usepackage{upquote}}{}
\IfFileExists{microtype.sty}{% use microtype if available
  \usepackage[]{microtype}
  \UseMicrotypeSet[protrusion]{basicmath} % disable protrusion for tt fonts
}{}
\makeatletter
\@ifundefined{KOMAClassName}{% if non-KOMA class
  \IfFileExists{parskip.sty}{%
    \usepackage{parskip}
  }{% else
    \setlength{\parindent}{0pt}
    \setlength{\parskip}{6pt plus 2pt minus 1pt}}
}{% if KOMA class
  \KOMAoptions{parskip=half}}
\makeatother
\usepackage{xcolor}
\usepackage[margin=1in]{geometry}
\usepackage{color}
\usepackage{fancyvrb}
\newcommand{\VerbBar}{|}
\newcommand{\VERB}{\Verb[commandchars=\\\{\}]}
\DefineVerbatimEnvironment{Highlighting}{Verbatim}{commandchars=\\\{\}}
% Add ',fontsize=\small' for more characters per line
\usepackage{framed}
\definecolor{shadecolor}{RGB}{248,248,248}
\newenvironment{Shaded}{\begin{snugshade}}{\end{snugshade}}
\newcommand{\AlertTok}[1]{\textcolor[rgb]{0.94,0.16,0.16}{#1}}
\newcommand{\AnnotationTok}[1]{\textcolor[rgb]{0.56,0.35,0.01}{\textbf{\textit{#1}}}}
\newcommand{\AttributeTok}[1]{\textcolor[rgb]{0.13,0.29,0.53}{#1}}
\newcommand{\BaseNTok}[1]{\textcolor[rgb]{0.00,0.00,0.81}{#1}}
\newcommand{\BuiltInTok}[1]{#1}
\newcommand{\CharTok}[1]{\textcolor[rgb]{0.31,0.60,0.02}{#1}}
\newcommand{\CommentTok}[1]{\textcolor[rgb]{0.56,0.35,0.01}{\textit{#1}}}
\newcommand{\CommentVarTok}[1]{\textcolor[rgb]{0.56,0.35,0.01}{\textbf{\textit{#1}}}}
\newcommand{\ConstantTok}[1]{\textcolor[rgb]{0.56,0.35,0.01}{#1}}
\newcommand{\ControlFlowTok}[1]{\textcolor[rgb]{0.13,0.29,0.53}{\textbf{#1}}}
\newcommand{\DataTypeTok}[1]{\textcolor[rgb]{0.13,0.29,0.53}{#1}}
\newcommand{\DecValTok}[1]{\textcolor[rgb]{0.00,0.00,0.81}{#1}}
\newcommand{\DocumentationTok}[1]{\textcolor[rgb]{0.56,0.35,0.01}{\textbf{\textit{#1}}}}
\newcommand{\ErrorTok}[1]{\textcolor[rgb]{0.64,0.00,0.00}{\textbf{#1}}}
\newcommand{\ExtensionTok}[1]{#1}
\newcommand{\FloatTok}[1]{\textcolor[rgb]{0.00,0.00,0.81}{#1}}
\newcommand{\FunctionTok}[1]{\textcolor[rgb]{0.13,0.29,0.53}{\textbf{#1}}}
\newcommand{\ImportTok}[1]{#1}
\newcommand{\InformationTok}[1]{\textcolor[rgb]{0.56,0.35,0.01}{\textbf{\textit{#1}}}}
\newcommand{\KeywordTok}[1]{\textcolor[rgb]{0.13,0.29,0.53}{\textbf{#1}}}
\newcommand{\NormalTok}[1]{#1}
\newcommand{\OperatorTok}[1]{\textcolor[rgb]{0.81,0.36,0.00}{\textbf{#1}}}
\newcommand{\OtherTok}[1]{\textcolor[rgb]{0.56,0.35,0.01}{#1}}
\newcommand{\PreprocessorTok}[1]{\textcolor[rgb]{0.56,0.35,0.01}{\textit{#1}}}
\newcommand{\RegionMarkerTok}[1]{#1}
\newcommand{\SpecialCharTok}[1]{\textcolor[rgb]{0.81,0.36,0.00}{\textbf{#1}}}
\newcommand{\SpecialStringTok}[1]{\textcolor[rgb]{0.31,0.60,0.02}{#1}}
\newcommand{\StringTok}[1]{\textcolor[rgb]{0.31,0.60,0.02}{#1}}
\newcommand{\VariableTok}[1]{\textcolor[rgb]{0.00,0.00,0.00}{#1}}
\newcommand{\VerbatimStringTok}[1]{\textcolor[rgb]{0.31,0.60,0.02}{#1}}
\newcommand{\WarningTok}[1]{\textcolor[rgb]{0.56,0.35,0.01}{\textbf{\textit{#1}}}}
\usepackage{graphicx}
\makeatletter
\def\maxwidth{\ifdim\Gin@nat@width>\linewidth\linewidth\else\Gin@nat@width\fi}
\def\maxheight{\ifdim\Gin@nat@height>\textheight\textheight\else\Gin@nat@height\fi}
\makeatother
% Scale images if necessary, so that they will not overflow the page
% margins by default, and it is still possible to overwrite the defaults
% using explicit options in \includegraphics[width, height, ...]{}
\setkeys{Gin}{width=\maxwidth,height=\maxheight,keepaspectratio}
% Set default figure placement to htbp
\makeatletter
\def\fps@figure{htbp}
\makeatother
\setlength{\emergencystretch}{3em} % prevent overfull lines
\providecommand{\tightlist}{%
  \setlength{\itemsep}{0pt}\setlength{\parskip}{0pt}}
\setcounter{secnumdepth}{-\maxdimen} % remove section numbering
\ifLuaTeX
  \usepackage{selnolig}  % disable illegal ligatures
\fi
\usepackage{bookmark}
\IfFileExists{xurl.sty}{\usepackage{xurl}}{} % add URL line breaks if available
\urlstyle{same}
\hypersetup{
  pdftitle={Lab 3},
  hidelinks,
  pdfcreator={LaTeX via pandoc}}

\title{Lab 3}
\author{}
\date{\vspace{-2.5em}Spring 2025}

\begin{document}
\maketitle

\section{Variable transformations}\label{variable-transformations}

The World Bank provides valuable data on a number of public health and
economic indicators for countries across the globe\footnote{You can
  access the data at \url{http://data.worldbank.org/}}. Today, we will
be looking indicators which might predict infant mortality, which is the
number of children (per 1000 births) who die before the age of 1.

\subsubsection{Questions}\label{questions}

\begin{itemize}
\tightlist
\item
  What factors do you think might affect or correlate with infant
  mortality?
\end{itemize}

In particular, we will be looking at 2 specific factors which might
correlate well with infant mortality (measured in 2015) - GDP per capita
(roughly how much income does the average individual produce) as
measured in 2013 and the proportion of the population with access to
electricity (as measured in 2012). I have removed countries which were
missing data for any of the variables.

\begin{Shaded}
\begin{Highlighting}[]
\NormalTok{fileName }\OtherTok{\textless{}{-}} \StringTok{"https://raw.githubusercontent.com/ysamwang/btry6020\_sp22/main/lab2/world\_bank\_data.csv"}
\NormalTok{wb.data }\OtherTok{\textless{}{-}} \FunctionTok{read.csv}\NormalTok{(fileName)}
\FunctionTok{head}\NormalTok{(wb.data)}
\end{Highlighting}
\end{Shaded}

\begin{verbatim}
##                country  elec_acc inf_mort gdp_capita
## 1              Andorra 100.00000      2.1 42806.5226
## 2          Afghanistan  43.00000     66.3   666.7951
## 3               Angola  37.00000     96.0  5900.5296
## 4              Albania 100.00000     12.5  4411.2582
## 5 United Arab Emirates  97.69783      5.9 42831.0891
## 6            Argentina  99.80000     11.1 14443.0657
\end{verbatim}

\subsubsection{Questions}\label{questions-1}

\begin{itemize}
\tightlist
\item
  What direction do you think the association is between each of these
  variables?
\item
  What strength do you think the association is between each of these
  variables?
\end{itemize}

We can use the \texttt{pairs} command to plot the many pairs of
variables at once. Note that we've excluded the first column here, since
that's just the name of countries

\begin{Shaded}
\begin{Highlighting}[]
\FunctionTok{pairs}\NormalTok{(wb.data[, }\SpecialCharTok{{-}}\DecValTok{1}\NormalTok{])}
\end{Highlighting}
\end{Shaded}

\includegraphics{lab3_files/figure-latex/unnamed-chunk-2-1.pdf}

\subsubsection{Questions}\label{questions-2}

\begin{itemize}
\tightlist
\item
  Does this look like what you might expect?
\item
  What sticks out?
\item
  Do the relationships look linear?
\end{itemize}

The relationship between electricity and infant mortality looks roughly
linear, but the relationship between GDP per capita and infant mortality
does not. Let's see how we might transform the data. The \texttt{log}
function by default returns the natural log (base e). Let's plot a few
transformations and see what makes the relationship linear.

\begin{Shaded}
\begin{Highlighting}[]
\CommentTok{\# using the par(mfrow = c(r, c)) puts multiple }
\CommentTok{\# plots together. The plots are arranged so }
\CommentTok{\# that there are r rows and c columns}

\FunctionTok{par}\NormalTok{(}\AttributeTok{mfrow =} \FunctionTok{c}\NormalTok{(}\DecValTok{2}\NormalTok{,}\DecValTok{2}\NormalTok{))}

\CommentTok{\# first argument is the X variable, second argument is the Y variable}
\CommentTok{\# main specifies the title, xlab specifies the x axis label }
\CommentTok{\# and ylab specifies the y axis label}
\FunctionTok{plot}\NormalTok{(wb.data}\SpecialCharTok{$}\NormalTok{gdp\_capita, wb.data}\SpecialCharTok{$}\NormalTok{inf\_mort, }\AttributeTok{main =} \StringTok{"Untransformed"}\NormalTok{,}
     \AttributeTok{xlab =} \StringTok{"gdp per capita"}\NormalTok{, }\AttributeTok{ylab =} \StringTok{"Infant Mortality (per 1000)"}\NormalTok{)}

\FunctionTok{plot}\NormalTok{(wb.data}\SpecialCharTok{$}\NormalTok{gdp\_capita, }\FunctionTok{log}\NormalTok{(wb.data}\SpecialCharTok{$}\NormalTok{inf\_mort),}
     \AttributeTok{main =} \StringTok{"log(mortality) \textasciitilde{} gdp/capita"}\NormalTok{,}
     \AttributeTok{xlab =} \StringTok{"gdp per capita"}\NormalTok{, }\AttributeTok{ylab =} \StringTok{"log(mortality)"}\NormalTok{)}

\FunctionTok{plot}\NormalTok{(}\FunctionTok{log}\NormalTok{(wb.data}\SpecialCharTok{$}\NormalTok{gdp\_capita), wb.data}\SpecialCharTok{$}\NormalTok{inf\_mort,}
     \AttributeTok{main =} \StringTok{"mortality \textasciitilde{} log(gdp/capita)"}\NormalTok{,}
     \AttributeTok{xlab =} \StringTok{"log(gdp per capita)"}\NormalTok{, }\AttributeTok{ylab =} \StringTok{"mortality"}\NormalTok{)}

\FunctionTok{plot}\NormalTok{(}\FunctionTok{log}\NormalTok{(wb.data}\SpecialCharTok{$}\NormalTok{gdp\_capita), }\FunctionTok{log}\NormalTok{(wb.data}\SpecialCharTok{$}\NormalTok{inf\_mort),}
     \AttributeTok{main =} \StringTok{"log(mortality) \textasciitilde{} log(gdp/capita)"}\NormalTok{,}
     \AttributeTok{xlab =} \StringTok{"log(gdp/capita)"}\NormalTok{, }\AttributeTok{ylab =} \StringTok{"log(mortality)"}\NormalTok{)}
\end{Highlighting}
\end{Shaded}

\includegraphics{lab3_files/figure-latex/unnamed-chunk-3-1.pdf}

The plots correspond to the models:
\[E(\text{mortality} \mid \text{gdp/capita}) = b_0 + b_1\text{gdp/capita}\]
\[E(\log(\text{mortality}) \mid \text{gdp/capita}) = b_0 + b_1\text{gdp/capita}\]
\[E(\text{mortality} \mid \text{gdp/capita}) = b_0 + b_1\log(\text{gdp/capita})\]
\[E(\log(\text{mortality}) \mid \text{gdp/capita}) = b_0 + b_1\log(\text{gdp/capita})\]

\subsubsection{Questions}\label{questions-3}

\begin{itemize}
\tightlist
\item
  Which transformation looks most linear?
\item
  How do we interpret the \(b_1\) parameter in each model?
\end{itemize}

The transformation that looks most linear takes the log of both
mortality and gdp per capita. We can estimate the transformed and
untransformed models now using the \texttt{lm} command.

\begin{Shaded}
\begin{Highlighting}[]
\CommentTok{\# Untransformed data}
\NormalTok{untransformed.reg }\OtherTok{\textless{}{-}} \FunctionTok{lm}\NormalTok{(inf\_mort }\SpecialCharTok{\textasciitilde{}}\NormalTok{ gdp\_capita, }\AttributeTok{data =}\NormalTok{ wb.data)}

\FunctionTok{summary}\NormalTok{(untransformed.reg)}
\end{Highlighting}
\end{Shaded}

\begin{verbatim}
## 
## Call:
## lm(formula = inf_mort ~ gdp_capita, data = wb.data)
## 
## Residuals:
##     Min      1Q  Median      3Q     Max 
## -24.011 -14.633  -5.749   8.625  67.583 
## 
## Coefficients:
##               Estimate Std. Error t value Pr(>|t|)    
## (Intercept)  3.168e+01  1.743e+00  18.171  < 2e-16 ***
## gdp_capita  -5.523e-04  7.093e-05  -7.787 5.68e-13 ***
## ---
## Signif. codes:  0 '***' 0.001 '**' 0.01 '*' 0.05 '.' 0.1 ' ' 1
## 
## Residual standard error: 19.07 on 176 degrees of freedom
## Multiple R-squared:  0.2562, Adjusted R-squared:  0.252 
## F-statistic: 60.63 on 1 and 176 DF,  p-value: 5.678e-13
\end{verbatim}

\begin{Shaded}
\begin{Highlighting}[]
\CommentTok{\# regression with transformed data}
\NormalTok{transformed.reg }\OtherTok{\textless{}{-}} \FunctionTok{lm}\NormalTok{(}\FunctionTok{log}\NormalTok{(inf\_mort) }\SpecialCharTok{\textasciitilde{}} \FunctionTok{log}\NormalTok{(gdp\_capita), }\AttributeTok{data =}\NormalTok{ wb.data)}

\FunctionTok{summary}\NormalTok{(transformed.reg)}
\end{Highlighting}
\end{Shaded}

\begin{verbatim}
## 
## Call:
## lm(formula = log(inf_mort) ~ log(gdp_capita), data = wb.data)
## 
## Residuals:
##      Min       1Q   Median       3Q      Max 
## -1.24132 -0.34865 -0.00525  0.34525  2.40377 
## 
## Coefficients:
##                 Estimate Std. Error t value Pr(>|t|)    
## (Intercept)      8.11682    0.24882   32.62   <2e-16 ***
## log(gdp_capita) -0.63135    0.02848  -22.17   <2e-16 ***
## ---
## Signif. codes:  0 '***' 0.001 '**' 0.01 '*' 0.05 '.' 0.1 ' ' 1
## 
## Residual standard error: 0.5554 on 176 degrees of freedom
## Multiple R-squared:  0.7363, Adjusted R-squared:  0.7348 
## F-statistic: 491.3 on 1 and 176 DF,  p-value: < 2.2e-16
\end{verbatim}

We can also look at the residuals plotted against fitted values and
fitted values vs observed values for both models. What does this suggest
about how each model fits our data?

\begin{Shaded}
\begin{Highlighting}[]
\FunctionTok{par}\NormalTok{(}\AttributeTok{mfrow =} \FunctionTok{c}\NormalTok{(}\DecValTok{1}\NormalTok{,}\DecValTok{2}\NormalTok{))}
\FunctionTok{plot}\NormalTok{(untransformed.reg}\SpecialCharTok{$}\NormalTok{fitted.values, untransformed.reg}\SpecialCharTok{$}\NormalTok{residuals, }\AttributeTok{main =} \StringTok{"Untransformed"}\NormalTok{,}
     \AttributeTok{xlab =} \StringTok{"fitted values"}\NormalTok{, }\AttributeTok{ylab =} \StringTok{"residuals"}\NormalTok{)}
\FunctionTok{plot}\NormalTok{(transformed.reg}\SpecialCharTok{$}\NormalTok{fitted.values, transformed.reg}\SpecialCharTok{$}\NormalTok{residuals, }\AttributeTok{main =} \StringTok{"Transformed"}\NormalTok{,}
     \AttributeTok{xlab =} \StringTok{"fitted values"}\NormalTok{, }\AttributeTok{ylab =} \StringTok{"residuals"}\NormalTok{)}
\end{Highlighting}
\end{Shaded}

\includegraphics{lab3_files/figure-latex/unnamed-chunk-5-1.pdf}

\begin{Shaded}
\begin{Highlighting}[]
\FunctionTok{par}\NormalTok{(}\AttributeTok{mfrow =} \FunctionTok{c}\NormalTok{(}\DecValTok{1}\NormalTok{,}\DecValTok{2}\NormalTok{))}
\FunctionTok{plot}\NormalTok{(untransformed.reg}\SpecialCharTok{$}\NormalTok{fitted.values, wb.data}\SpecialCharTok{$}\NormalTok{inf\_mort, }\AttributeTok{main =} \StringTok{"Untransformed"}\NormalTok{,}
     \AttributeTok{xlab =} \StringTok{"fitted values"}\NormalTok{, }\AttributeTok{ylab =} \StringTok{"Obs Values"}\NormalTok{)}


\NormalTok{fitted.values.log }\OtherTok{\textless{}{-}} \FunctionTok{exp}\NormalTok{(transformed.reg}\SpecialCharTok{$}\NormalTok{fitted.values }\SpecialCharTok{+} \FunctionTok{summary}\NormalTok{(transformed.reg)}\SpecialCharTok{$}\NormalTok{sigma}\SpecialCharTok{\^{}}\DecValTok{2}\SpecialCharTok{/}\DecValTok{2}\NormalTok{)}
\FunctionTok{plot}\NormalTok{(fitted.values.log,wb.data}\SpecialCharTok{$}\NormalTok{inf\_mort, }\AttributeTok{main =} \StringTok{"Transformed"}\NormalTok{,}
     \AttributeTok{xlab =} \StringTok{"fitted values"}\NormalTok{, }\AttributeTok{ylab =} \StringTok{"Obs Values"}\NormalTok{)}
\end{Highlighting}
\end{Shaded}

\includegraphics{lab3_files/figure-latex/unnamed-chunk-6-1.pdf}

\subsubsection{Questions}\label{questions-4}

\begin{itemize}
\tightlist
\item
  What do you notice about the fitted values for the untransformed data?
  Hint: What is the range of fitted values, and does it make sense given
  the variable we are predicting?
\item
  Compare the \(R^2\) from both regressions. What does this suggest
  about which explanatory variable is a better predictor of infant
  mortality?
\item
  Why do you think this is true?
\item
  Note that we aren't exactly comparing apples to apples here because
  one regression has log(mortality) as the response while the other uses
  mortality untransformed. Is there a way you could make the comparison
  more fair?
\item
  Which model would you use if you are trying to predict infant
  mortality for a country not in the data set? Which model would you use
  if you are trying to explain to a collaborator? Which model would you
  use if you are trying to test if infant mortality is associated with
  gdp/capita?
\item
  Repeat the exercise but with electricity access? Which model would you
  select when using electricity access? What about when you include both
  electricity access and gdp per capita?
\end{itemize}

\newpage

\subsection{Housing Data}\label{housing-data}

In class, we've been discussing data about housing prices and in last
week's lab, we considered modeling the home prices with polynomial
regression. As a quick refresher, recall that there are 522 observations
with the following variables:

\begin{itemize}
\tightlist
\item
  price: in 2002 dollars
\item
  area: Square footage
\item
  bed: number of bedrooms
\item
  bath: number of bathrooms
\item
  ac: central AC (yes/no)
\item
  garage: number of garage spaces
\item
  pool: yes/no
\item
  year: year of construction
\item
  quality: high/medium/low
\item
  home style: coded 1 through 7
\item
  lot size: sq ft
\item
  highway: near a highway (yes/no)
\end{itemize}

\begin{Shaded}
\begin{Highlighting}[]
\NormalTok{fileName }\OtherTok{\textless{}{-}} \FunctionTok{url}\NormalTok{(}\StringTok{"https://raw.githubusercontent.com/ysamwang/btry6020\_sp22/main/lectureData/estate.csv"}\NormalTok{)}
\NormalTok{housing\_data }\OtherTok{\textless{}{-}} \FunctionTok{read.csv}\NormalTok{(fileName)}

\FunctionTok{head}\NormalTok{(housing\_data)}
\end{Highlighting}
\end{Shaded}

\begin{verbatim}
##   id  price area bed bath  ac garage pool year quality style   lot highway
## 1  1 360000 3032   4    4 yes      2   no 1972  medium     1 22221      no
## 2  2 340000 2058   4    2 yes      2   no 1976  medium     1 22912      no
## 3  3 250000 1780   4    3 yes      2   no 1980  medium     1 21345      no
## 4  4 205500 1638   4    2 yes      2   no 1963  medium     1 17342      no
## 5  5 275500 2196   4    3 yes      2   no 1968  medium     7 21786      no
## 6  6 248000 1966   4    3 yes      5  yes 1972  medium     1 18902      no
\end{verbatim}

\begin{Shaded}
\begin{Highlighting}[]
\NormalTok{housing\_data}\SpecialCharTok{$}\NormalTok{age }\OtherTok{\textless{}{-}} \DecValTok{2002} \SpecialCharTok{{-}}\NormalTok{ housing\_data}\SpecialCharTok{$}\NormalTok{year}
\end{Highlighting}
\end{Shaded}

\subsection{Categorical variables}\label{categorical-variables}

In our data, Housing Style is coded 1 through 7

\begin{Shaded}
\begin{Highlighting}[]
\FunctionTok{table}\NormalTok{(housing\_data}\SpecialCharTok{$}\NormalTok{style)}
\end{Highlighting}
\end{Shaded}

\begin{verbatim}
## 
##   1   2   3   4   5   6   7   9  10  11 
## 214  58  64  11  18  18 136   1   1   1
\end{verbatim}

In class, we described how to include categorical variables in a
regression by picking a reference category and then including binary
variables for the other categories. R does this entire process for us
inside the lm command.

\begin{Shaded}
\begin{Highlighting}[]
\DocumentationTok{\#\#\#}
\CommentTok{\# Include style}
\NormalTok{model1 }\OtherTok{\textless{}{-}} \FunctionTok{lm}\NormalTok{(price }\SpecialCharTok{\textasciitilde{}}\NormalTok{ area }\SpecialCharTok{+}\NormalTok{ style, }\AttributeTok{data =}\NormalTok{ housing\_data)}
\FunctionTok{summary}\NormalTok{(model1)}
\end{Highlighting}
\end{Shaded}

\begin{verbatim}
## 
## Call:
## lm(formula = price ~ area + style, data = housing_data)
## 
## Residuals:
##     Min      1Q  Median      3Q     Max 
## -271624  -34852   -5465   28660  312589 
## 
## Coefficients:
##               Estimate Std. Error t value Pr(>|t|)    
## (Intercept) -1.030e+05  1.126e+04  -9.152  < 2e-16 ***
## area         1.875e+02  5.857e+00  32.021  < 2e-16 ***
## style       -1.286e+04  1.625e+03  -7.912 1.54e-14 ***
## ---
## Signif. codes:  0 '***' 0.001 '**' 0.01 '*' 0.05 '.' 0.1 ' ' 1
## 
## Residual standard error: 74820 on 519 degrees of freedom
## Multiple R-squared:  0.7069, Adjusted R-squared:  0.7058 
## F-statistic: 625.8 on 2 and 519 DF,  p-value: < 2.2e-16
\end{verbatim}

\begin{Shaded}
\begin{Highlighting}[]
\DocumentationTok{\#\#\#}
\CommentTok{\# Include style as a factor (i.e., make sure R knows it is categorical data)}
\NormalTok{model2 }\OtherTok{\textless{}{-}} \FunctionTok{lm}\NormalTok{(price }\SpecialCharTok{\textasciitilde{}}\NormalTok{ area }\SpecialCharTok{+} \FunctionTok{as.factor}\NormalTok{(style), }\AttributeTok{data =}\NormalTok{ housing\_data)}
\FunctionTok{summary}\NormalTok{(model2)}
\end{Highlighting}
\end{Shaded}

\begin{verbatim}
## 
## Call:
## lm(formula = price ~ area + as.factor(style), data = housing_data)
## 
## Residuals:
##     Min      1Q  Median      3Q     Max 
## -273461  -34602   -4571   28259  310176 
## 
## Coefficients:
##                      Estimate Std. Error t value Pr(>|t|)    
## (Intercept)        -1.162e+05  1.286e+04  -9.034  < 2e-16 ***
## area                1.882e+02  6.094e+00  30.886  < 2e-16 ***
## as.factor(style)2  -2.040e+04  1.107e+04  -1.843   0.0659 .  
## as.factor(style)3  -1.785e+04  1.066e+04  -1.674   0.0948 .  
## as.factor(style)4  -3.446e+04  2.311e+04  -1.491   0.1366    
## as.factor(style)5  -8.499e+04  1.856e+04  -4.578 5.90e-06 ***
## as.factor(style)6  -7.597e+04  1.867e+04  -4.068 5.49e-05 ***
## as.factor(style)7  -7.854e+04  1.043e+04  -7.528 2.35e-13 ***
## as.factor(style)9   2.033e+04  7.504e+04   0.271   0.7866    
## as.factor(style)10 -8.684e+04  7.597e+04  -1.143   0.2535    
## as.factor(style)11 -6.179e+04  7.493e+04  -0.825   0.4100    
## ---
## Signif. codes:  0 '***' 0.001 '**' 0.01 '*' 0.05 '.' 0.1 ' ' 1
## 
## Residual standard error: 74750 on 511 degrees of freedom
## Multiple R-squared:  0.7119, Adjusted R-squared:  0.7063 
## F-statistic: 126.3 on 10 and 511 DF,  p-value: < 2.2e-16
\end{verbatim}

\subsubsection{Questions}\label{questions-5}

\begin{itemize}
\tightlist
\item
  What is the reference category that R is using?
\item
  How would you interpret the estimated coefficients?
\item
  What is the estimated difference in home price when comparing a house
  which is style 2 against a house which is style 4?
\end{itemize}

\subsection{Interaction terms}\label{interaction-terms}

Last week, we examined how home prices were associated with age and
modeled the relationship with polynomial regressions. If you recall,
none of the models fit particularly well. Turns out, using a log
transformation on housing price seems to make the relationship more
linear.

\begin{Shaded}
\begin{Highlighting}[]
\NormalTok{model3 }\OtherTok{\textless{}{-}} \FunctionTok{lm}\NormalTok{(}\FunctionTok{log}\NormalTok{(price) }\SpecialCharTok{\textasciitilde{}}\NormalTok{ age, }\AttributeTok{data =}\NormalTok{ housing\_data)}

\FunctionTok{par}\NormalTok{(}\AttributeTok{mfrow =} \FunctionTok{c}\NormalTok{(}\DecValTok{1}\NormalTok{,}\DecValTok{2}\NormalTok{))}
\FunctionTok{plot}\NormalTok{(model3}\SpecialCharTok{$}\NormalTok{fitted.values, }\FunctionTok{log}\NormalTok{(housing\_data}\SpecialCharTok{$}\NormalTok{price), }\AttributeTok{xlab =} \StringTok{"Fitted Values"}\NormalTok{, }\AttributeTok{ylab =} \StringTok{"Obs Values"}\NormalTok{)}
\FunctionTok{plot}\NormalTok{(model3}\SpecialCharTok{$}\NormalTok{fitted.values, model3}\SpecialCharTok{$}\NormalTok{residuals, }\AttributeTok{xlab =} \StringTok{"Fitted Values"}\NormalTok{, }\AttributeTok{ylab =} \StringTok{"Residuals"}\NormalTok{)}
\end{Highlighting}
\end{Shaded}

\includegraphics{lab3_files/figure-latex/unnamed-chunk-10-1.pdf}

We see from the estimated coefficients that an older home is typically
less expensive than a newer home.

\begin{Shaded}
\begin{Highlighting}[]
\FunctionTok{summary}\NormalTok{(model3)}
\end{Highlighting}
\end{Shaded}

\begin{verbatim}
## 
## Call:
## lm(formula = log(price) ~ age, data = housing_data)
## 
## Residuals:
##      Min       1Q   Median       3Q      Max 
## -0.86899 -0.24789 -0.07036  0.22367  1.46833 
## 
## Coefficients:
##               Estimate Std. Error t value Pr(>|t|)    
## (Intercept) 12.9356873  0.0342333  377.87   <2e-16 ***
## age         -0.0142770  0.0008717  -16.38   <2e-16 ***
## ---
## Signif. codes:  0 '***' 0.001 '**' 0.01 '*' 0.05 '.' 0.1 ' ' 1
## 
## Residual standard error: 0.3509 on 520 degrees of freedom
## Multiple R-squared:  0.3403, Adjusted R-squared:  0.339 
## F-statistic: 268.2 on 1 and 520 DF,  p-value: < 2.2e-16
\end{verbatim}

However, as we discussed in class, we might also expect that the
association of price and age depends on the quality of the home. We can
fit a model with the interaction between age and quality to see

\begin{Shaded}
\begin{Highlighting}[]
\CommentTok{\# We can include each covariate and the interaction term in the lm formula}
\NormalTok{model4 }\OtherTok{\textless{}{-}} \FunctionTok{lm}\NormalTok{(}\FunctionTok{log}\NormalTok{(price) }\SpecialCharTok{\textasciitilde{}}\NormalTok{ age }\SpecialCharTok{+}\NormalTok{ quality }\SpecialCharTok{+}\NormalTok{ age }\SpecialCharTok{*}\NormalTok{ quality, }\AttributeTok{data =}\NormalTok{ housing\_data)}
\FunctionTok{summary}\NormalTok{(model4)}
\end{Highlighting}
\end{Shaded}

\begin{verbatim}
## 
## Call:
## lm(formula = log(price) ~ age + quality + age * quality, data = housing_data)
## 
## Residuals:
##      Min       1Q   Median       3Q      Max 
## -0.70937 -0.16798 -0.00132  0.14146  0.82006 
## 
## Coefficients:
##                     Estimate Std. Error t value Pr(>|t|)    
## (Intercept)       13.2543198  0.0518526 255.615   <2e-16 ***
## age               -0.0045991  0.0026210  -1.755   0.0799 .  
## qualitylow        -1.0931907  0.0888987 -12.297   <2e-16 ***
## qualitymedium     -0.6222432  0.0631323  -9.856   <2e-16 ***
## age:qualitylow     0.0023796  0.0029750   0.800   0.4241    
## age:qualitymedium -0.0003452  0.0028204  -0.122   0.9026    
## ---
## Signif. codes:  0 '***' 0.001 '**' 0.01 '*' 0.05 '.' 0.1 ' ' 1
## 
## Residual standard error: 0.2524 on 516 degrees of freedom
## Multiple R-squared:  0.6615, Adjusted R-squared:  0.6582 
## F-statistic: 201.7 on 5 and 516 DF,  p-value: < 2.2e-16
\end{verbatim}

\begin{Shaded}
\begin{Highlighting}[]
\CommentTok{\# Alternatively, if we only explicitly specify the interaction term, the main}
\CommentTok{\# effects are automatically included}
\NormalTok{model5 }\OtherTok{\textless{}{-}} \FunctionTok{lm}\NormalTok{(}\FunctionTok{log}\NormalTok{(price) }\SpecialCharTok{\textasciitilde{}}\NormalTok{ age }\SpecialCharTok{*}\NormalTok{ quality, }\AttributeTok{data =}\NormalTok{ housing\_data)}
\FunctionTok{summary}\NormalTok{(model5)}
\end{Highlighting}
\end{Shaded}

\begin{verbatim}
## 
## Call:
## lm(formula = log(price) ~ age * quality, data = housing_data)
## 
## Residuals:
##      Min       1Q   Median       3Q      Max 
## -0.70937 -0.16798 -0.00132  0.14146  0.82006 
## 
## Coefficients:
##                     Estimate Std. Error t value Pr(>|t|)    
## (Intercept)       13.2543198  0.0518526 255.615   <2e-16 ***
## age               -0.0045991  0.0026210  -1.755   0.0799 .  
## qualitylow        -1.0931907  0.0888987 -12.297   <2e-16 ***
## qualitymedium     -0.6222432  0.0631323  -9.856   <2e-16 ***
## age:qualitylow     0.0023796  0.0029750   0.800   0.4241    
## age:qualitymedium -0.0003452  0.0028204  -0.122   0.9026    
## ---
## Signif. codes:  0 '***' 0.001 '**' 0.01 '*' 0.05 '.' 0.1 ' ' 1
## 
## Residual standard error: 0.2524 on 516 degrees of freedom
## Multiple R-squared:  0.6615, Adjusted R-squared:  0.6582 
## F-statistic: 201.7 on 5 and 516 DF,  p-value: < 2.2e-16
\end{verbatim}

\subsubsection{Questions}\label{questions-6}

\begin{itemize}
\tightlist
\item
  Write out the form of the model that is being estimated
\item
  Looking at the estimated coefficients, are you surprised by the
  results?
\item
  Do you think the relationship between age and price differs depending
  on quality?
\item
  What are some reasons you might include the interaction term in your
  model?
\item
  What are some reasons you might choose to not include the interaction
  term in your model?
\end{itemize}

\end{document}
